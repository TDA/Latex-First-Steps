\documentclass[a4paper,12pt,titlepage]{article} % seems like it takes only 10, 11 and 12 :D
\author{Sai Prashanth Chandramouli}
\title{First Steps}
\makeindex
\pagestyle{plain} % can be headings, plain, or empty
% sets the headers and the footers basically
% now i see why it is so much easier to use latex, no need to set headers footers with menu stuff :D
\begin{document}
	% this makes the title
	\maketitle
	% look how it generates the title with title, author name and date
	% this auto generates a toc :D
	\tableofcontents
	% this is a section
	First Example, so, Hello World?
	\section{First Section}
		This is the first section of my doc.\\
		Double slashes are newlines
		\subsection[Sub Short]{Sub Section}
			Lets create a nice subsection here
	\section[Short Title]{Second Section}	
		This is the second section of my doc.\\
		White space is like rendered as in html, in\\
		other words, ignored.
		
	\paragraph{Hey}
		This is a short paragraph that is automatically justified and hyphenated by \LaTeX{} for us :) Isn't that
		 simply wonderful?? We don't even have to worry about how the page looks cause that is \LaTeX's work to do!! Today's date can also be generated in a similar way, just insert backslash	
		\linebreak[3] followed by 'today', and it will display the date. Be sure to check whether today is \today.
	\pagebreak
	\paragraph
		This para is for showing dots: ... and ellipsis \ldots
		
	\paragraph{}
		This is for \emph{emphasis}. \emph{But if you are emphasizing something inside an \emph{emphasis}, it changes to normal}
		
	\begin{enumerate}
		\item Sai Pc
		\begin{itemize}
			\item Is awesome
			\item Is cute
			\item Is brainy
		\end{itemize}
		\item Annnnnnd
	\end{enumerate}
	
	\begin{flushleft}
		This text is left aligned
	\end{flushleft}

	\begin{flushright}
		This text is right aligned
	\end{flushright}		
	
	\begin{center}
		This text is centered
	\end{center}		
	
	This is a verse:
	\begin{verse}
		This is going to be,\\
		Useful for my poems
	\end{verse}
	
%	\chapter[TOC]{Table Of Contents}
%		hello

	\begin{verbatim}
		This is like the pre tag in html
		you can have \ldots and other such stuff, and itll print normally
		including newlines etc
	\end{verbatim}
	
	\begin{abstract}
		This is an abstract with default styles applied :)
	\end{abstract}
		
\end{document}